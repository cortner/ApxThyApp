% !TeX root = ./apxthy.tex


\section{Least Squares Methods}

\subsection{Motivation}
%
We first describe least squares methods in abstract terms. Let 
$[a,b]$ be an interval and 
$b_1, \dots, b_N \in C([a,b])$ be $N$ linearly independent basis functions 
for an approximation space 
\[
    \AA_N   := {\rm span}\b\{ b_1, \dots, b_N \b\}. 
\]
Given $w \in C(a,b) \cap L^1(a,b)$ (note the open interval!) we can 
define a weighted $L^2$-inner product 
\[
    \< f, g \>_{L^2_w} := \int_a^b w(x) f(x) g(x)^* \,dx,    
\]
with associated norm $\|f\|_{L^2_w} := \<f,f\>_{L^2_w}^{1/2}$. The best
approximation of a function $f \in C([a,b])$ with respect to this weighted norm
is then given by 
\begin{equation} \label{eq:lsq:ctslsq}
    p_N \in \min_{p \in \AA_N} \b\| f - p \b\|_{L^2_w}^2.
\end{equation}
We call this a continuous least squares problem. 

Computationally, we typically need to discretise \eqref{eq:lsq:ctslsq}. 
To that end, we choose points $x_1, \dots, x_M \in [a, b]$ and weights 
$w_1, \dots w_M$ and define the discrete inner product 
\begin{equation} \label{eq:lsq:disip}
    \< f, g \>_{\ell^2_w} := \sum_{m = 1}^M w_m f(x_m) g(x_m)^*
\end{equation}
with associated norm $\|f\|_{\ell^2_w} := \<f,f \>_{\ell^2_w}^{1/2}$. This 
gives the discrete least squares problem 
\begin{equation}
    \label{eq:lsq:dislsq}
    p_N \in \min_{p \in \AA_N} \b\| f - p \b\|_{\ell_w^2}.
\end{equation}
This is the typical kind of least squares problem encountered in 
real applications. 

We distinguish two scenarios:
\begin{enumerate}
\item {\bf User Chooses Data: } In this scenario the ``user'' is given a
function $f$ to be approximated. She may then choose the points $x_m$, weights
$w_m$ and evaluations $f(x_m)$ in order to fine-tune and optimise the fit $p_N$.
For example it is then feasible to start from \eqref{eq:lsq:ctslsq} and design a
discrete LSQ system \eqref{eq:lsq:dislsq} that approximates
\eqref{eq:lsq:ctslsq} in a suitable sense. An arbtirary amount of data $(x_m,
f(x_m))$ may then be generated to ensure a good fit.

\item {\bf Data is provided: } Some data has been collected outside the control 
of the person (``user'') designing the fit. Given the data points $(x_m, f(x_m))$ 
(possibly subject to noise, i.e. $y_m = f(x_m) + \eta_m$ is then provided)
one then needs to choose an appropriate approximation space $\AA_N$, 
approximation degree $N$ and weights $w_m$ to ensure a good fit in a sense 
dictated by the application. 
\end{enumerate}

We will study both scenarios but note that the second one is the more 
typical in applications. 

\subsection{Solution methods}
%
\label{sec:lsq:soln}
%
We convert \eqref{eq:lsq:dislsq} into a linear algebra problem. 
By writing 
\[
    Y_m := f(x_m),  \sqrt{W} := {\rm diag}(\sqrt{w_1}, \dots, \sqrt{w_m}) \in \R^{M \times M}
\]
and 
\[
    p(x_m) = \sum_{n = 1}^N c_n b_n(x_m) = A c, 
    \quad \text{where }  A_{mn} = b_n(x_m),
\]
then $A \in \R^{M \times N}$ and we obtain 
\[
    \sum_{m = 1}^M w_m | p(w_m) - f(x_m)|^2 
    = \big\| \sqrt{W} A c - \sqrt{W} Y \big\|^2,
\]
where $\|\cdot\|$ denotes the standard Euclidean norm in $\R^M$. 
We write $\tilde{A} := \sqrt{W} A, \tilde{Y} := \sqrt{W} Y$ and 
write the least squares functional equivalently as 
\[
    \Phi(c) := \big\| \sqrt{W} A c - \sqrt{W} Y \big\|^2 
    = c^T \tilde{A}^T \tilde{A} c - 2 c^T \tilde{A}^T \tilde{Y} + \|\tilde{Y}\|^2.
\]
A minimiser must satisfy $\nabla\Phi(c) = 0$, which gives the linear system 
\begin{equation} \label{eq:lsq:normaleqns}
    \tilde{A}^T\tilde{A} c = \tilde{A}^T \tilde{Y}.
\end{equation}
These are called the normal equations, which can be solved using 
the LU or Cholesky factorisation. 

It turns out that they are often (though not always) ill-conditioned. 
An alternative approach is therefore to perform the (numerically stable)
{\em thin QR factorisation}
\[
    \tilde{A} = Q R,   
\]
where $R \in \R^{N \times N}$ is upper triangular and  $Q \in \R^{M \times N}$
has ortho-normal columns, i.e., $Q^T Q = I \in \R^{N \times N}$. 
With the QR factorisation in hand the normal equations can be rewritten as 
\begin{align*}
    \tilde{A}^T\tilde{A} c &= \tilde{A}^T \tilde{Y} \\ 
    \Leftrightarrow \qquad 
    R^T Q^T Q R c &= R^T Q^T \tilde{Y} \\ 
    \Leftrightarrow \qquad 
    R c &= Q^T \tilde{Y},
\end{align*}
provided that $R$ is invertible (which is equivalent to $A^T A$ being invertible 
and to $A$ having full rank). Thus, the solution of the least squares problem 
becomes 
\begin{equation}
    \label{eq:lsq:qr}
    R c = Q^T \sqrt{W} Y, \qquad \text{where} \qquad 
    \sqrt{W} A = QR.
\end{equation}

It is worthwhile comparing the computational cost of the two approaches. 
\begin{enumerate}
\item The assembly of thenormal equations requires the multiplication 
$A^T A$ which requires $O(M N^2)$ operations, followed by  the 
Cholesky factorisation of $A^T A$ which requires $O(N^3)$ operiations. 
Thus the cost of solving \eqref{eq:lsq:normaleqns} is $O(M N^2)$. 
\item The cost of the QR factorisation in \eqref{eq:lsq:qr} is 
$O(M N^2)$ as well, while the inversion of $R$ is only $O(N^2)$ and the 
multiplication with $Q^T$ is $O(NM)$. 

Thus both algorithms scale like $O(M N^2)$.
\end{enumerate}


\subsection{Orthogonal Polynomials}
%
\label{sec:lsq:orthpolys}
%


\subsection{Accuracy and Stability I: Least Squares and Nodal Interpolation}
%
Consider fitting trigonometric polynomials $\AA_{2N} = \TT_N'$ with equispaced
grid points $x_n = \pi n / N$ and uniform weights $w_n = 1$. Then the
least squares fit 
\[
    \min \sum_{n = 0}^{2N-1} |f(x_n) - t(x_n)|^2 
\]
is equivalent to trigonometric interpolation, for which we have sharp and error
estimates that predict a close to optimal rate of approximation. 

We could leave it at this, but it is still interesting to observe what happens 
to the least squares system. The matrix $A$ is now given by 
\[
    A_{nk} = e^{ikx_n}
\]
and the entries in the normal equation by 
\[
    [A^* A]_{kk'} = \sum_{n} e^{-ikx_n} e^{ik' x_n} = 2 N \delta_{kk'}
\]
according to Exercise~\ref{exr:trig:trapezoidal rule}(i). 
This is due to the fact that the discrete inner product 
\eqref{eq:lsq:disip}  (up to a constant factor) identical to 
the $L^2(\TT)$-inner product on the space $\TT_N'$, that is, 
\[
    \< f, g \>_{\ell^2} = 2N \mint_{-\pi}^\pi f g^* \,dx \qquad 
    \forall f, g \in \TT_N'.
\]
No QR factorisation is needed and the lsq fit is given by 
\[
    c = (2 N)^{-1} A^* Y,  
\]
where the operation $(2N)^{-1} A^T Y$ can be performed at $O(N \log N)$
computational cost using the FFT. 

Analogous observations are of course true for connecting least squares 
methods and algebraic polynomials. 


\subsection{Accuracy and Stability II: Random data}
%
\label{sec:lsq:rand}
%
The situation gets more interesting when we are not allowed to optimise the
points at which to fit the approximant. There is an infinite variety of
different situations that can occur when the provided data is application
driven, which goes far beyond the scope of this module. Here, we will assume
that the points $x_m$ are distributed according to some probability law. That
is, they are random. This is in fact a rather common situation in applications
as well. Note also that we are now in the Case-2 situtation where we are given a
fixed amount of data $(x_m, f(x_m))_{m = 1}^M$ and should choose $N$ ensure the
best possible fit given the data we have. In particular this means that we
should not choose $N$ too large!

Specifically, we shall assume throughout this section that 
\begin{equation}
    \label{eq:lsw:wm_law}
    x_m \sim w \,dx, \qquad \text{are iid}, \quad \text{for } m = 1, \dots, M.
\end{equation}
(identically and independently distributed) and without loss of generality that
$\int_a^b w \,dx = 1$ as this can always be achieved by rescaling. We also
assume that $w \in C(a,b) \cap L^1(a,b)$ as before. In this case, we can
construct a sequence of orthogonal polynomials as in \S~\ref{sec:lsq:orthpolys}
with respect to the $L^2_w$-inner product and we will target best approximation 
with respect to the same inner product. 

We will discuss two fundamental results due to Cohen, Davenport and Leviatan
\cite{Cohen2013-yj}, but we won't prove them.  The first result concerns the
{\em stability} of the normal equations. Specifically, we will show that if we
use an $L^2_w$-orthogonal basis then $A^T A$ will be close to identity (and in
particular invertible) for a sufficiently large number of sample points $x_m$. 



\begin{theorem}[Stability] \label{th:lsq:randstab}
    Let $\phi_1, \dots, \phi_N$ be $L^2_w$-orthonormal, $A_{mk} := \phi_k(x_m)$, 
    then 
    \[
        \mathbb{P}\B[ \| A^* A - I \|_{\rm op} > \delta\B]
        \leq 2 N \exp\B( - \smfrac{C_\delta M}{K(N)}\B), 
    \]
    where $C_\delta = (1+\delta) \log (1+\delta) - \delta$ and 
    \[
        K(N) = \sup_{x \in [a,b]} \sum_{k = 1}^N |\phi_k(x)|^2.
    \]
\end{theorem}

Let us specifically focus on the Chebyshev measure where $w(x) =
(1-x^2)^{-1/2}$, the Chebyshev basis $T_k(x)$ on the interval $[a,b]= [-1,1]$.
Since $|T_k(x)| \leq 1$ it readily follows that $K(N) \leq N$. Moreover, the
recursion formulat for $T_k$ implies that $T_k(1) = 1$ for all $k$, hence this
bound is sharp, i.e., $K(N) = N$ in this case. 

To make $N \exp( - \smfrac{C_\delta M}{N})$ small, we therefore need 
to choose $N \leq \alpha M / \log M$. With this choice,
\[
    N \exp( - \smfrac{C_\delta M}{N})
    = 
    N \exp\b( - \alpha C_\delta \log M\b)
    \leq 
    M^{1 -\alpha C_\delta} / \log M
\]
and by choosing $\alpha$ sufficiently large we can ensure that this value tends
to zero as $M \to \infty$. (the case of sufficiently large amounts of data).
Conversely, if $\alpha$ is too small, then $M^{1 -\alpha C_\delta} / \log M \to
\infty$ as $M \to \infty$ which shows that the choice $N \leq \alpha M/\log M$
is sharp. This is a very mild restriction!

The next result  we discuss concerns the approximation $p_{NM} = \sum_n c_n
\phi_n$ we obtain by solving the least squares problem $\| f - p_{NM}
\|_{\ell^2(\{x_m\})}^2 \to \min$.

\begin{theorem} \label{th:lsq:randerr}
    There exists a constant $c$ such that, if 
    \[
        K(N) \leq \frac{c}{1+r} \frac{M}{\log M},
    \]
    then, 
    \[
        \mathbb{E}\b[ \|f - p_{NM}\|_{L^2_w}^2 \b] 
        \leq 
        (1+\epsilon(M)) \|f - \Pi_N f\|_{L^2_w}^2 
        + 8 \|f\|_{L^\infty(a,b)}^2 M^{-r},
    \]  
    where $\epsilon(M) \to 0$ as $M \to \infty$, and $\Pi_N$ denotes the
    best-approximation operator with respect to the $L^2_w$-norm.
\end{theorem}

As a first comment, we observe that our restriction $N \leq \alpha M / \log M$
for sufficiently small $\alpha$ re-appears. (In fact, this prerequisite is
required to be able to apply Theorem~\ref{th:lsq:randstab}.)

We can now ask what consequence the choice $N = \alpha M / \log M$ has 
on the error. In this case, $\alpha = c / (1+r)$, or equivalently, 
$r = c/\alpha - 1$ hence (sacrifycing just a log-factor)
\[
    M^{-r}  \leq N^{-r} = N^{1 - c/\alpha} =: N^{-\alpha'},
\]
where $\alpha' > 0$ provided that $\alpha$ is chosen sufficiently small. In this
case, we can conclude that 
\[
    \mathbb{E}\b[ \|f - p_{NM}\|_{L^2_w}^2 \b] 
    \lesssim  
    \|f - \Pi_N f\|_{L^2_w}^2 
    + 
    N^{- \alpha'}
\]
for some $\alpha' > 0$. Thus, for differentiable functions $f$, such a choice is
quasi-optimal. 

However, for analytic functions the rate in the error estimate is
reduced. Let us assume that $f \in A(E_\rho)$, then 
$\| f - \Pi_N f \|_{L^2_w} \lesssim \rho^{-N}$ hence we must balance the 
two constributions 
\[
    \rho^{-N} + M^{-r}.
\]
We have already seen that $N = \alpha M/\log M$ leads to $\rho^{-N} \ll M^{-r}$, 
hence we instead attempt to choose $N = a (M/\log M)^{\alpha}$ for some $0 < \alpha < 1$,
which gives 
\[
    r = c' (M / \log M)^{1-\alpha}.
\]
Thus, we wish to balance 
\begin{align*}
    \exp\B[ - (\log \rho) N\B]  &+ \exp\B[ - r \log M \B]  \\ 
    = \exp\B[ - c'' M^\alpha (\log M)^{-\alpha} \B]  &+ \exp\B[ - c M^{1-\alpha} (\log M)^{-\alpha} \B] 
\end{align*}
We can now see that the two terms are balanced when $\alpha = 1/2$, that is, 
the quasi-optimal choice of $N$ appears to be 
\[
    N = a \b(M / \log M\b)^{1/2}.
\]
This is also somewhat consistent with the observation that in the $C^j$ case we
need to decrease $\alpha$ for increasing $j$. 

That said, we should remember that we have balanced an error estimate and not
the actual error. At this point, it is highly advisable to test these
predictions numerically, which is done in \nblsq, where we see --- for some
limited examples --- that the stability condition $N \lesssim M/\log M$ appears
to be crucial but the stronger requirement $N \lesssim (M/\log M)^{1/2}$ seems
to not be required.

In summary, the foregoing analysis is intended to demonstrate how different
competing contributions to approximation by fitting from data can be balanced at
least in principle but also the limitations of analysis.





\subsection{Exercises}
%
\label{sec:lsq:exercises}
%